\documentclass {report}

\usepackage[magyar]{babel}
\usepackage[utf8]{inputenc}
\usepackage{t1enc}

\begin{document}
\title{Egylépéses gyártási feladatok költségoptimális ütemezése időzített automatával}

\author{Vida Judit\\Témavezető: Dr. Hegyháti Máté\\Széchenyi István Egyetem}
\maketitle


\tableofcontents
\chapter{Bevezetés}
Mivel foglalkozunk,  miért, hogy fogtunk hozzá, milyen eredményt értünk el.
Miertnek a resze, hogy masok mit csinaltak, mi miben vagyunk masok.
korabbi eredmenyek ismertetese (kulonbozo modszerekkel) - kik foglalkoztak vele\\ problema bemutatasa
szoveges tartalomjegyzek. (nem most, hanem majd ha kesz).
masfel oldal kb.

\chapter{Irodalmi áttekintés}
Utemezesi feladatok altalaban, gyartasi feladatok, publikalt modszerek, algoritmusok.\\ \\
Ütemezési feladatok az élet számos területén megjelennek, hiszen nagyon sok olyan élethelyzet adódik, amikor egy adott problémára meg kell találnunk a lehető legjobb, legoptimálisabb megoldást. Ezek a problémák adódhatnak mindennapi szituációkból, mint a napi feladataink felosztása, vagy ebben az esetben megtalálni a legoptimálisabb megoldást egy specifikus problémára. \\Közös bennük, hogy adottak a feladatok, a feladatokat elvégezni képes berendezések vagy eszközök, egy adott időintervallum minden feladathoz, a cél pedig, hogy a kijelölt feltételek mellett kielégítsük a végcélt. Az egyes feladatok  intervalluma az az  időtartam, ami a munka elvégzéséhez szükséges, ez általában minden esetben egyedien meghatározott, tehát minden különböző feladatnak eltérő mennyiségű időre van szüksége. A feladatoknak lehet elvégzési határideje is, amit szintén figyelembe kell venni az ütemezés folyamán.  \\
Az elvégzendő feladatokat a szaknyelvben gyakran hívják ezen kívül munkáknak vagy  tevékenységeknek, az erőforrások pedig sok esetben gépek, amiket felszereléseknek vagy berendezéseknek neveznek. \\ \\ offline, online, semi-offline\\ \\
Az offline típusú problémáknál minden szükséges bemeneti adat elérhető az ütemezési feladat megkezdésekor, ezzel szemben az online típusnál a döntéseket úgy kell meghozni, hogy a munka paraméterei az ütemezés kezdetekor nem elérhetők. A semi-offline a két kategória között helyezkedik el, a feladat néhány paramétere adott a döntések meghozása előtt. \\
Ezen kívül megkülönböztetjük a sztochaikus problémákat, ahol a paraméterek futtatáskor vesznek fel értéket, az összes többi esetben a feladatot determinisztikusnak nevezzük. A két kategóriát viszont nem lehet élesen elhatárolni egymástól, hiszen vannak olyan sztochaikus feladatok, amiket semi-offline-ként lehet definiálni, \mbox{mivel néhány paraméterre szükség van az ütemezés megkezdése előtt.} 

\section{Alap ütemezési problémák}
 A vegyiparban a következőknél sokkal bonyolultabb, több paraméterrel rendelkező problémákat kell megoldani, ezek azonban a komplexebb feladatok alapját képezik. 
 \subsection{}
 
 \section{Vegyipari folyamatok ütemezése}
 A vegyipari feladatok ütemezése komplexebb, mint az alap ütemezési problémáké, de azért sok tekintetben hasonló elveken alapul. \\ \\Az ebbe a feladatkörbe tartozó problémáknál a következő adatok adottak:
 
 \begin{itemize}
 \item  a recept
  \item a tárolási kapacitás
  \item a független és a körülményektől függő paraméterek
  \item  a kiegészítő paraméterek
 \end{itemize}
 
 \section{Feltevések}
 Ha nincsenek egyénileg meghatározva, az alábbi feltevések érvényesülnek:
 \begin{itemize}
 \item Egyedi helykiosztás\\Minden munka egy berendezéshez kerül társításra, még akkor is, ha több másik berendezés is el tudná végezni a feladatot
 \item Nem preemtív\\A feladatot nem lehet megszakítani, hogy egy másikkal folytatódjon az ütemezés egy berendezésen belül
 \item Kötegelt (szakaszos) eljárások\\Ha egy feladat folyamatos, ...........................
 \item Meghatározott méretek\\A feldogozott anyag mennyisége nem haladhatja meg az előre adott mértéket. 
  \end{itemize}
 \section{Recept}
 Az ütemezésnél három egységet különböztetünk meg.\\
 \textbf{Termékek} az ábécé nagybetűivel jelöltek. Általában A, B, C, vagy P1, P2, P3 a Product megnevezés miatt.\\
 \textbf{Feladatok} szintén nagybetűvel jelöltek az alapján, hogy melyik termékek tartoznak hozzájuk (Pl. A1, A2, A3), de néhány helyen alkalmazzák a T1 jelölést is, matematikai képletekben pedig alsó indexben i karaktert a feladatok jelölésére.\\ \\
 \textbf{Berendezések} leginkább U1, U2, U3... jelöléssel jelenik meg a Unit megfelelője miatt. Ritkábban, de használatos E karakterrel is, matematikai képletekben pedig j alsó indexként.\\
  

4-5 oldal
sok hivatkozas


\chapter{Probléma definíció}
Konkretan a sajat problemaosztalyunkat, lehetoleg formalisan is.
vegere: nomenclatura
kis pelda.
1-2 hivatkozas.
2-3 oldal

\chapter{Időzített automaták, UPPAAL}
\section{Idozitett automata} 1 oldal
\section{UPPAAL-CORA} 2-3 oldal

\chapter{LPTA alapu single stage utemezes}
    \section{description}
    \section{template}
    \section{problem def es query}

\chapter{Teszteredmenyek, osszehasonlitas}
tablazatok

\chapter{Osszefoglalas}

\chapter*{Hivatkozasok}
bibtex

\appendix

\chapter{Jelolesek}



\end{document}
