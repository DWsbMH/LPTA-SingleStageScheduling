\documentclass {report}

\usepackage[magyar]{babel}
\usepackage[utf8]{inputenc}
\usepackage{t1enc}

\begin{document}
\title{Egylépéses gyártási feladatok költségoptimális ütemezése időzített automatával}

\author{Vida Judit\\Témavezető: Dr. Hegyháti Máté\\Széchenyi István Egyetem}
\maketitle


\tableofcontents
\chapter{Bevezetés}
Mivel foglalkozunk,  miért, hogy fogtunk hozzá, milyen eredményt értünk el.
Miertnek a resze, hogy masok mit csinaltak, mi miben vagyunk masok.
korabbi eredmenyek ismertetese (kulonbozo modszerekkel) - kik foglalkoztak vele\\ problema bemutatasa
szoveges tartalomjegyzek. (nem most, hanem majd ha kesz).
masfel oldal kb.

\chapter{Irodalmi áttekintés}
Utemezesi feladatok altalaban, gyartasi feladatok, publikalt modszerek, algoritmusok.
4-5 oldal
sok hivatkozas


\chapter{Probléma definíció}
Konkretan a sajat problemaosztalyunkat, lehetoleg formalisan is.
vegere: nomenclatura
kis pelda.
1-2 hivatkozas.
2-3 oldal

\chapter{Időzített automaták, UPPAAL}
\section{Idozitett automata} 1 oldal
\section{UPPAAL-CORA} 2-3 oldal

\chapter{LPTA alapu single stage utemezes}
    \section{description}
    \section{template}
    \section{problem def es query}

\chapter{Teszteredmenyek, osszehasonlitas}
tablazatok

\chapter{Osszefoglalas}

\chapter*{Hivatkozasok}
bibtex

\appendix

\chapter{Jelolesek}



\end{document}
