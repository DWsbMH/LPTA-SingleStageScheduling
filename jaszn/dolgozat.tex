\documentclass {report}

\usepackage[magyar]{babel}
\usepackage[utf8]{inputenc}
\usepackage{t1enc}

\begin{document}
\title{Egylépéses gyártási feladatok költségoptimális ütemezése időzített automatával}

\author{Vida Judit\\Témavezető: Dr. Hegyháti Máté\\Széchenyi István Egyetem}
\maketitle


\tableofcontents
\chapter{Bevezetés}
Mivel foglalkozunk,  miért, hogy fogtunk hozzá, milyen eredményt értünk el.
Miertnek a resze, hogy masok mit csinaltak, mi miben vagyunk masok.
korabbi eredmenyek ismertetese (kulonbozo modszerekkel) - kik foglalkoztak vele\\ problema bemutatasa
szoveges tartalomjegyzek. (nem most, hanem majd ha kesz).
masfel oldal kb.

\chapter{Irodalmi áttekintés}
Utemezesi feladatok altalaban, gyartasi feladatok, publikalt modszerek, algoritmusok.\\ \\
Ütemezési feladatok az élet számos területén megjelennek, hiszen nagyon sok olyan élethelyzet adódik, amikor egy adott problémára meg kell találnunk a lehető legjobb, legoptimálisabb megoldást. Ezek a problémák adódhatnak mindennapi szituációkból, mint a napi feladataink felosztása, vagy ebben az esetben megtalálni a legoptimálisabb megoldást egy specifikus problémára. \\Közös bennük, hogy adottak a feladatok, a feladatokat elvégezni képes berendezések vagy eszközök, egy adott időintervallum minden feladathoz, a cél pedig, hogy a kijelölt feltételek mellett kielégítsük a végcélt. Az egyes feladatok  intervalluma az az  időtartam, ami a munka elvégzéséhez szükséges, ez általában minden esetben egyedien meghatározott, tehát minden különböző feladatnak eltérő mennyiségű időre van szüksége. A feladatoknak lehet elvégzési határideje is, amit szintén figyelembe kell venni az ütemezés folyamán.  \\
Az elvégzendő feladatokat a szaknyelvben gyakran hívják ezen kívül munkáknak vagy  tevékenységeknek, az erőforrások pedig sok esetben gépek, amiket felszereléseknek vagy berendezéseknek neveznek. \\ \\ offline, online, semi-offline\\ \\
Az offline típusú problémáknál minden szükséges bemeneti adat elérhető az ütemezési feladat megkezdésekor, ezzel szemben az online típusnál a döntéseket úgy kell meghozni, hogy a munka paraméterei az ütemezés kezdetekor nem elérhetők. A semi-offline a két kategória között helyezkedik el, a feladat néhány paramétere adott a döntések meghozása előtt. \\
Ezen kívül megkülönböztetjük a sztochaikus problémákat, ahol a paraméterek futtatáskor vesznek fel értéket, az összes többi esetben a feladatot determinisztikusnak nevezzük. A két kategóriát viszont nem lehet élesen elhatárolni egymástól, hiszen vannak olyan sztochaikus feladatok, amiket semi-offline-ként lehet definiálni, \mbox{mivel néhány paraméterre szükség van az ütemezés megkezdése előtt.} 

\section{Alap ütemezési problémák}
 A vegyiparban a következőknél sokkal bonyolultabb, több paraméterrel rendelkező problémákat kell megoldani, ezek azonban a komplexebb feladatok alapját képezik. 
 \subsection{}
 
 \section{Vegyipari folyamatok ütemezése}
 A vegyipari feladatok ütemezése komplexebb, mint az alap ütemezési problémáké, de azért sok tekintetben hasonló elveken alapul. \\ \\Az ebbe a feladatkörbe tartozó problémáknál a következő adatok adottak:
 
 \begin{itemize}
 \item  a recept
  \item a tárolási kapacitás
  \item a független és a körülményektől függő paraméterek
  \item  a kiegészítő paraméterek
 \end{itemize}
 
 \section{Feltevések}
 Ha nincsenek egyénileg meghatározva, az alábbi feltevések érvényesülnek:
 \begin{itemize}
 \item Egyedi helykiosztás\\Minden munka egy berendezéshez kerül társításra, még akkor is, ha több másik berendezés is el tudná végezni a feladatot
 \item Nem preemtív\\A feladatot nem lehet megszakítani, hogy egy másikkal folytatódjon az ütemezés egy berendezésen belül
 \item Kötegelt (szakaszos) eljárások\\Ha egy feladat folyamatos, ...........................
 \item Meghatározott méretek\\A feldogozott anyag mennyisége nem haladhatja meg az előre adott mértéket. 
  \end{itemize}
 \section{Recept}
 Az ütemezésnél három egységet különböztetünk meg.\\
 \textbf{Termékek}\\
 \textbf{Feladatok} \\ \\
 \textbf{Berendezések} \\
  (melyiket mivel fogjuk jelölni)
  
  \section{Egyéb megoldó módszerek}
  Ebben a dolgozatban időzített automatákkal foglalkozunk a meghatározott probléma megoldására, de ezen felül számos más módszerrel lehet megközelíteni a gyártási vagy egyéb folyamatok optimalizálását. Ezekkel a módszerekkel már sikeresen megoldottak hasonló problémákat, ezeket az eddig publikált módszereket említjük meg röviden.
  \subsection{MILP modellek}
  A MILP modellek a legelterjedtebb megoldó módszerek közé tartoznak, és több altípusuk létezik.
  \subsubsection{Time discretization based}
  A módszer alapján időpontokat és időréseket különböztetünk meg. A időrés az az intervallum, amely egyik időponttól a következőig tart, az időpont pedig ennek a fordítottja, tehát az időrés kezdete az egyik időpont, az időrés vége pedig egy másik. \\
  A feladatokhoz bináris változókat rendelnek aszerint, hogy a feladat az adott időpontban elvégzésre kerül, vagy nem. Amikor a feladat abban az időpontban megvalósul, akkor 1 lesz a bináris változó értéke, ha nem, akkor 0. Így annyi bináris változóra lesz szükség, ahány időpontot meghatároztunk. Lehetőleg minél kisebb számú időpont felvételével kell megtalálni az optimális megoldást, ezért alkalmazzák a következő algoritmust.
  \\A lehető legkevesebb időponttal kezdik meg az optimalizálási folyamatot, de mivel az eredmény nem lesz pontos, minden egyes sikeres optimalizálás elvégzése után növelik az időpontok számát, amíg egyre jobb értéket kapnak. \\
  Az időfelosztásos modellek előnye, hogy széles skálán mozog a megoldható problémák típusa, ...
 Két altípust különböztetnek meg az alapján, hogy az időpontokat az optimalizálás előtt meghatározzák, ezek a Fix időpontos időfelosztásos módszerek, vagy pedig csak a feladat közben kerül meghatározásra az időpontok száma. Utóbbiakat Variable time model-eknek hívják, és a lényegük, hogy minél kevesebb bináris változóra legyen szükség. A modellekben folyamatos változókat használnak, amik meghatározzák mindegyik időponthoz tartozó feladatokat. 
 \\- rossz LP relaxáció (?)
 \\-big M
 \\-kevés bináris változó -> jobb CPU teljesítmény
  
  
 
  \subsubsection{Precedence based formulations (elsőbbségi alapú)}
  Jobb számítási hatékonysággal rendelkező módszer, mint az időfelosztásos, viszont nem alkalmazható olyan széles körben. Két típusú változó jelenik meg a modellben, mindegyik bináris.\\
  \begin{displaymath}
 Y_(ij):\end{displaymath} 	ahol i a feladat, j pedig a berendezés. Akkor ad 1 értéket, ha j berendezés elvégzi i feladatot \\ 
   \begin{displaymath}X_(iji')\end{displaymath}
   A változó értéke akkor 1, ha j berendezés i és i' feladatot is ellátja, valamint i-t korábban végzi el, mint i'-t.
   
  
  \subsection{Analysis based (elemzés alapú)}
  Az elemzés alapú formákat is széles körben használják az optimalizálásban, mert számos előnyük van a MILP modellekhez és az S-gráfokhoz képest, mégis összességében elmarad a hatékonyságuk az imént említett módszerekétől. \\Két legfontosabb fajtájuk az időzített Petri-háló és az időzített automata. A Petri hálókkal és automatákkal leginkább diszkrét típusú eseményeket modelleznek, a hatékonyság növelése miatt az alap modellek kiegészültek időzítéssel. 
 
  \subsection{S-gráf}
  Az első gráf alapú optimalizációra fejlesztett módszer, amely nemcsak vizuálisan szemlélteti a folyamatot, de egyben egy matematikai modell is. Irányított gráfokból áll, amelynek a csomópontjai feladatok és az azokon elvégzendő műveletek, amelyeket az élek kötnek össze őket.
  \subsubsection{Megoldó algoritmusok}
  
  

4-5 oldal
sok hivatkozas


\chapter{Probléma definíció}
Konkretan a sajat problemaosztalyunkat, lehetoleg formalisan is.
vegere: nomenclatura
kis pelda.
1-2 hivatkozas.
2-3 oldal

\chapter{Időzített automaták, UPPAAL}
\section{Idozitett automata} 1 oldal
\section{UPPAAL-CORA} 2-3 oldal

\chapter{LPTA alapu single stage utemezes}
    \section{description}
    \section{template}
    \section{problem def es query}

\chapter{Teszteredmenyek, osszehasonlitas}
tablazatok

\chapter{Osszefoglalas}

\chapter*{Hivatkozasok}
bibtex

\appendix

\chapter{Jelolesek}



\end{document}
